\documentclass{scrartcl}

\usepackage[hidelinks]{hyperref}
\usepackage[none]{hyphenat}
\usepackage{setspace}
\doublespace

%Please include a clear, concise, and descriptive title
\title{The Innovations of the Vectrex and how they were not Enough to Save it from Failure}

%Please do not change the subtitle
\subtitle{COMP130 - Game Platform History Essay}

%Please put your student ID in the author field
\author{HM184836}

\begin{document}

\maketitle

\abstract
{
This short essay explores the climate of the games industry in the early 1980s whilst introducing the Vectrex and exploring the innovations and unique properties that made it stand out from the other consoles of 2nd generation. It analyses the reasons for the Vectrex's failure despite its many innovations and discusses whether the Vectrex could have made a larger impact if it had lived a longer life before the video game crash took place.
}


\section*{Introduction}
In 1982, General Consumer Electronics released the Vectrex -- the first vector based home console. Like most other second generation consoles, it had an 8-bit microprocessor and read games from cartridges. However, it came with a built in vector display, rather than connecting to home televisions. This is one of several unique features discussed in this essay that make the Vectrex stand out. Despite being such an innovative product, the Vectrex failed to make much of a mark on the industry; unfortunately, the video game crash that occurred in 1983 cut its life short. Perhaps the Vectrex could have had a brighter future and made a bigger impression if it had been released several years earlier.


\section*{The Industry in the Early 1980s}

The video game market was turbulent at the time the Vectrex was released. There were many changes taking place in the industry, both in technology and business practice. This included the growing popularity of home computers; the beginning of third party game development; and a surge of companies trying to enter the video games industry as a result of its success\cite{wolf:pong}. The increasingly accessible market resulted in a surplus of games and consoles of questionable quality with little variation amongst them, forming the perfect environment for an industry crash \cite{ernkvist:crash}. The Vectrex was not just another console made by a company looking to take advantage of the lucrative video game market, though -- the thought put into its design and innovations alongside its high production cost demonstrate this\cite{}. Unfortunately, perhaps unfairly, it got lost amongst the many consoles being released and passed by relatively unnoticed.
%More successful in Europe


\section*{The Innovations of the Vectrex and their Impact}

\subsection*{Vector Graphics}
The distinguishing feature of the Vectrex was its vector graphics. The raster scan technology used in home CRT TVs at the time filled the screen by scanning the electron beam along horizontal rows down the screen, whereas vector graphics scanned the electron beam between points stored as coordinates to draw graphics from a series of line segments \cite{wolf:medium}. One advantage of vector graphics is that it makes it possible to produce the illusion of depth effectively due to the higher resolution\cite{perron:theory} and ability to easily perform transformations on the lines \cite{defanti:impact}. Thus the Vectrex, although not powerful enough to run a fully 3D game \cite{perron:theory}, could be seen as one of the first steps towards bringing 3D games to home consoles.

Unfortunately vector graphics were limited by the capabilities of hardware at the time and meant that Vectrex games were monochrome and couldn't display complex graphics and backgrounds\cite{}. The Vectrex's solution to this was to use plastic overlays that are placed on the screen, similar to the those used on the original Magnavox Odyssey. Some people may have found this unappealing, viewing it as a step backwards when compared to the colourful graphics of popular consoles at the time such as the Atari 2600 and ColecoVision. A colour version of the Vectrex was developed\cite{defanti:impact}, but never released. Jay Smith, the creator of the Vectrex emphasised in an interview in \textit{Classic Videogame Hardware Genius Guide}\cite{imagine:genius} that the colour version was expensive to produce and had some technical impracticalities. On top of this, the video game crash sealed the fate of the colour Vectrex. If it had made it to the market, perhaps this version would have appealed to a wider audience. Despite the aesthetic drawbacks, vector graphics do have a unique appearance and could have made the Vectrex stand out from the other consoles. 

Another consequence of being vector based was that Vectrex games were not able to be played on people's home TVs. Instead, the Vectrex came with its own built in vector CRT\cite{vectrex:manual}. This had positive implications for family homes, as the TV would not be occupied while somebody was playing the Vectrex. On the other hand, people may not have appreciated the bulk of the Vectrex's built in monitor, when consoles that could plug conveniently into the TV existed.
%In an interview, Hope Niemann said that this played a role in its success in europe\cite{kent:ultimate}.

\subsection*{Games}
The Vectrex's vector display meant that popular vector arcade games of the 1970s could be faithfully brought to the home. Although versions of vector arcade games had already been released on consoles such as the Atari 2600, limitations presented by raster graphics meant they had a different visual style\cite{}. The Vectrex, however, allowed these games to be played at home with the same familiar style used in the arcade versions. Many of the Vectrex's releases were ports of Cinematronic's arcade games such as \textit{Space Wars} and \textit{Star Castle} \cite{}, although there were some original releases too. It lacked popular titles such as \textit{Asteroids} (it came with a similar game built in called \textit{Mine Storm}, though), which may have reduced the appeal. Although people may have appreciated being able to play their favourite arcade games at home, the small library of games (only 28 games were officially released) available on the Vectrex, many being ports of arcade games, could have limited its popularity when people had the option of buying a different console that had many more games available. On the other hand, it was never given the opportunity to build a larger catalogue of games. 

After the Milton Bradley, who acquired the Vectrex in 1983, went out of business as a result of the video game crash, the rights reverted to Jay Smith, one of the creators of the Vectrex. He decided to release the Vectrex to the public domain for non-commercial purposes\cite{wolf:pong}. As a result of this, people started developing homebrew games (some even developing hardware\cite{madtronix}) for the Vectrex, with Vectrex games still being made in modern day. This shows that the Vectrex hasn't been completely forgotten and at least made an impact on people, if not the industry.


\subsection*{Peripherals} %May just delete this section
Additionally, two peripherals were released for the Vectrex - the 3D Imager and the Light Pen. The 3D Imager was a device for viewing specially designed games in stereoscopic 3D using a disc that spins in front of the eyes\cite{zachara:stereo}. This was the first 3D viewing device on a home games console \cite{}. The Light Pen allowed the user to draw directly onto the screen in the art and music programs released for it. Although the Vectrex was following the trend of the other console manufacturers by releasing accessories, the peripherals that were released were unique and innovative, with the 3D Imager being the first of its kind\cite{}. Unfortunately, both peripherals released for the Vectrex had incredibly short lives due to the crash, so it is difficult to gauge whether people would have perceived them as exciting new piece of technology or simply a gimmick.
%Both are now sought after by collectors\cite{kohler:hacks} wishing to own a rare piece of gaming history.


\section*{Conclusion}

The Vectrex brought a unique and interesting product to the market, but that was not enough to save it from  failure. The Vectrex may not have been able to make an impression on the industry, but it certainly made an impression on people - as can be seen by the dedicated community of developers still making homebrew games for it. Perhaps it would have fared better if the climate in the industry was more stable, as it could be argued that the Vectrex was never given a chance to become the best it could be. If the video game crash wasn't impending shortly after its release, the Vectrex may have been an innovative new product that got people excited. 


\bibliographystyle{ieeetr}
\bibliography{game-platform-history-references}

\end{document}



%The controller had 4 buttons and a self-centering analog joystick. A second controller can be connected to the console. \cite{vectrex:manual}

