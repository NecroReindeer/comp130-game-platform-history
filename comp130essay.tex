\documentclass{scrartcl}

\usepackage[hidelinks]{hyperref}
\usepackage[none]{hyphenat}
\usepackage{setspace}
\doublespace

%Please include a clear, concise, and descriptive title
\title{The Innovations of the Vectrex and how they were not Enough to Save it from the Crash}

%Please do not change the subtitle
\subtitle{COMP130 - Game Platform History Essay}

%Please put your student ID in the author field
\author{HM184836}


\begin{document}

\maketitle


\abstract
{
The Vectrex was a unique product that brought many innovations to home consoles. Unfortunately, the Vectrex's life was cut short by the video game crash that took place in 1983. The poor timing of its release combined with its high production cost meant that innovation was not enough to save it from the crash. Additionally, its innovations may instead have presented themselves as differences or gimmicks, making consumers wary. As a result, the Vectrex was unable make any significant impact on the video games industry. If the climate of the industry had been more stable, the Vectrex may have been able to be successful.

%This short essay explores the climate of the games industry in the early~1980s whilst introducing the Vectrex and exploring the innovations and unique properties that made it stand out from the other consoles of the second generation. It analyses the reasons for the Vectrex's failure despite its many innovations and discusses whether the Vectrex could have made a larger impact if it had lived a longer life before the video game crash took place.
}



\section*{Introduction}
In 1982, General Consumer Electronics released the Vectrex, the first vector-based home console. Like most other second generation consoles, it had an~8-bit microprocessor and read games from cartridges~\cite{vectrex:manual}. The Vectrex attempted to compete with the other consoles of the generation using its likeness to arcade systems and the unique properties of its technology and peripherals~\cite{vectrex:advert}. Despite being an innovative product, the Vectrex failed to make much of a mark on the industry; unfortunately, the video game crash that occurred in~1983 cut its life short. Perhaps the Vectrex could have had a brighter future if it had been released several years earlier and given more time to develop.

% Despite offering the comparatively powerful~\cite[p. 84]{tomczak:music} Motorola 68A09 microprocessor~\cite{vectrex:manual}, the Vectrex mainly competed using its likeness to arcade systems and unique properties of its technology and peripherals, as demonstrated by its advertisements~\cite{vectrex:advert}.


\section*{The Industry in the Early 1980s}
The video game market was turbulent in the early~1980s. Many changes were taking place in the industry, including the beginning of third party game development and a surge of companies trying to enter the successful video games industry~\cite[p. 58]{wolf:pong}. The increasingly accessible market resulted in a surplus of games and consoles of questionable quality with little variation amongst them. This formed the perfect environment for an industry crash~\cite{ernkvist:crash}. The Vectrex was not just another console made by a company looking to take advantage of the lucrative video game market, however; the thought put into its design and innovations demonstrate this. Unfortunately, perhaps unfairly, it got lost amongst the many consoles being released and passed by unnoticed. The Vectrex did see a small amount of success initially, though this was most likely aided by being competitively priced at~\$199~\cite[p. 233]{kent:ultimate}. However, the Vectrex had a high production cost due to its vector display, which resulted in its downfall when the crash forced games and consoles to be sold off cheaply~\cite[p. 289]{baer:supercade}.



\section*{The Innovations of the Vectrex}


\subsection*{Vector Graphics}
The distinguishing feature of the Vectrex was its use of vector technology. The raster scan technology used in home CRT televisions at the time filled the screen by scanning the electron beam in horizontal rows down the screen, whereas vector CRTs scanned the electron beam between points stored as coordinates, creating graphics as a series of line segments~\cite[pp. 20--21]{wolf:medium}. Vector graphics provided higher resolution~\cite[pp. 155--156]{perron:theory} and the ability to easily perform transformations on lines, making it possible to produce the illusion of depth effectively and perform sophisticated animations efficiently~\cite{defanti:impact}. Thus the Vectrex, though not powerful enough to run a fully 3D game~\cite[pp. 155--156]{perron:theory}, could be seen as one of the first steps towards bringing 3D games to home consoles.

As a consequence of using vector graphics, Vectrex games were unable to be played on home televisions. Instead, the Vectrex came with its own built in vector CRT~\cite{vectrex:manual}. This had positive implications for family homes, as the television would not be occupied by the Vectrex. On the other hand, people may not have appreciated the bulk of the Vectrex's monitor, when consoles existed that could plug conveniently into the television.

Furthermore, vector graphics were limited by the capabilities of the hardware at the time. This meant that Vectrex games were monochrome and could not display complex graphics. Its solution to this was to use plastic overlays placed on the screen, similar to the those used with the original Magnavox Odyssey~\cite[p. 70]{wolf:pong}. Some people may have found this unappealing, viewing it as a step backwards when compared to the colourful graphics of the Atari~2600 and ColecoVision. A colour version of the Vectrex was developed~\cite{defanti:impact}, but never saw release due to the crash. Perhaps this version would have appealed to a wider audience and facilitated the Vectrex's success. Despite the aesthetic drawbacks, vector graphics have a unique appearance and would have made the Vectrex stand out amongst the raster-based consoles. 


\subsection*{Games}
In addition to allowing games to exhibit smooth animations and the illusion of depth, the Vectrex's vector display meant that the popular vector arcade games of the~1970s could be played at home. Although versions of vector arcade games had already been released for the Atari~2600, the limitations presented by raster graphics meant that they had a different visual style~\cite{montfort:random}. The Vectrex, however, presented these games with the same familiar style of the arcade versions. While many of its releases were ports of arcade games, it lacked popular titles such as \textit{Asteroids} (although, it did have a similar game, \textit{Mine Storm}, built in~\cite[p. 289]{baer:supercade}), which may have reduced the appeal. Although people may have appreciated being able to play their favourite arcade games at home, the Vectrex's small library of games could have limited its popularity when people had the option of buying a different console with a larger selection of games available. On the other hand, it was never given the opportunity to build a larger catalogue of games. 


\subsection*{Peripherals} 
Additionally, the Vectrex attempted to innovate in the arena of peripherals. Two were released: the~3D Imager, a device predating the SegaScope that allowed specially designed games to be viewed in stereoscopic~3D~\cite{zachara:stereo}; and the Light Pen, which allowed the user to draw directly onto the screen in art and music games. Although the Vectrex was following the trend of the other console manufacturers by releasing accessories, its peripherals were unique and innovative in comparison to the keyboards and alternative controllers released for the Atari~2600~\cite[pp. 24, 62]{montfort:beam}. Unfortunately, both peripherals released for the Vectrex had incredibly short lives due to the crash, so it is difficult to gauge whether people would have perceived them as an exciting new piece of technology or simply a gimmick.

% It also offered features built in that other consoles required a purchase for, such as speech synthesis\cite{a thing}, which was offered on the Intellivision through the Intellivoice module. 



\section*{Conclusion}
The Vectrex's innovations were not enough to save it from the crash. Its slim profit margin combined with the poor timing of its release hindered its success. Moreover, its uniqueness may have worked against it, making some consumers wary of its differences and unfamiliar presentation. Perhaps it would have fared better if the climate in the industry was more stable, as it could be argued that the Vectrex was never given a chance to develop and mature. If the crash was not impending at the time of its release, the Vectrex may instead have excited people.


\bibliographystyle{ieeetran}
\bibliography{game-platform-history-references}


\end{document}
