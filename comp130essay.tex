\documentclass{scrartcl}

\usepackage[hidelinks]{hyperref}
\usepackage[none]{hyphenat}
\usepackage{setspace}
\doublespace

%Please include a clear, concise, and descriptive title
\title{The Innovations of the Vectrex and how they were not Enough to Save it from the Crash}

%Please do not change the subtitle
\subtitle{COMP130 - Game Platform History Essay}

%Please put your student ID in the author field
\author{HM184836}


\begin{document}

\maketitle


\abstract
{
This short essay explores the climate of the games industry in the early 1980s whilst introducing the Vectrex and exploring the innovations and unique properties that made it stand out from the other consoles of 2nd generation. It analyses the reasons for the Vectrex's failure despite its many innovations and discusses whether the Vectrex could have made a larger impact if it had lived a longer life before the video game crash took place.
}



\section*{Introduction}
In 1982, General Consumer Electronics released the Vectrex -- the first vector based home console. Like most other second generation consoles, it had an 8-bit microprocessor and read games from cartridges. However, it came with a built in vertical vector display, rather than connecting to home televisions. This is one of several unique features that made the Vectrex stand out. Despite being such an innovative product, the Vectrex failed to make much of a mark on the industry; unfortunately, the video game crash that occurred in 1983 cut its life short. Perhaps the Vectrex could have had a brighter future if it had been released several years earlier and given more time to develop.



\section*{The Industry in the Early 1980s}
The video game market was turbulent in the early 1980s. Many changes were taking part in the industry, including the beginning of third party game development and a surge of companies trying to enter the successful video games industry\cite{wolf:pong}. The increasingly accessible market resulted in a surplus of games and consoles of questionable quality with little variation amongst them, forming the perfect environment for an industry crash \cite{ernkvist:crash}. The Vectrex was not just another console made by a company looking to take advantage of the lucrative video game market, though -- the thought put into its design and innovations demonstrate this. Unfortunately, perhaps unfairly, it got lost amongst the many consoles being released and passed by unnoticed. Additionally, the competitively priced \$200\cite{defanti:impact} Vectrex had a high production cost due to its vector display, meaning that its profit margin would have been slim. Although this would have helped sales, it resulted huge losses when the crash forced games and consoles to be sold off cheaply\cite{baer:supercade}.



\section*{The Innovations of the Vectrex and their Impact}


\subsection*{Vector Graphics}
The distinguishing feature of the Vectrex was its vector graphics. The raster scan technology used in home CRT TVs at the time filled the screen by scanning the electron beam in horizontal rows down the screen, whereas vector graphics scanned the electron beam between points stored as coordinates, creating a series of line segments \cite{wolf:medium}. Vector graphics provide higher resolution\cite{perron:theory} and the ability to easily perform transformations on lines, making it possible produce the illusion of depth effectively and perform sophisticated animations efficiently\cite{defanti:impact}. Thus the Vectrex, although not powerful enough to run a fully 3D game\cite{perron:theory}, could be seen as one of the first steps towards bringing 3D games to home consoles.

Unfortunately vector graphics were limited by the capabilities of hardware at the time, meaning that Vectrex games were monochrome and couldn't display complex graphics. Its solution to this was to use plastic overlays placed on the screen\cite{wolf:pong}, similar to the those used on the original Magnavox Odyssey. Some people may have found this unappealing, viewing it as a step backwards when compared to the colourful graphics of consoles such as the Atari 2600 and ColecoVision. A colour version of the Vectrex was developed\cite{defanti:impact} but never saw release due to its production cost and technical impracticalities, according to Jay Smith, the Vectrex's creator, in an interview in \textit{Classic Videogame Hardware Genius Guide}\cite{imagine:genius}, in addition to the crash. If it had been more viable and made it to the market, perhaps this version would have appealed to a wider audience and facilitated the Vectrex's success. Despite the aesthetic drawbacks, vector graphics have a unique appearance and would have made the Vectrex stand out. 

Another consequence of being vector based was that Vectrex games were unable to be played on people's home TVs. Instead, the Vectrex came with its own built in vector CRT\cite{vectrex:manual}. This had positive implications for family homes, as the TV would not be occupied by the Vectrex. On the other hand, people may not have appreciated the bulk of the Vectrex's monitor, when consoles that could plug conveniently into the TV existed.


\subsection*{Games}
The Vectrex's vector display meant that popular vector arcade games of the 1970s could be brought to the home. Although versions of vector arcade games had already been released on consoles such as the Atari 2600, limitations presented by raster graphics meant that they had a different style\cite{montfort:random}. The Vectrex, however, allowed these games to be played at home with the same familiar style of the arcade versions. While many of its releases were ports of arcade games, it lacked popular titles such as \textit{Asteroids} (although, it had a similar game, \textit{Mine Storm}, built in)\cite{baer:supercade}, which may have reduced the appeal. Although people may have appreciated being able to play their favourite arcade games at home, its small library of games could have limited its popularity when people had the option of buying a different console that had a larger selection of games available. On the other hand, it was never given the opportunity to build a larger catalogue of games. 


\subsection*{Peripherals} 
Some of the Vectrex's games were for use with the two peripherals that were released: the 3D Imager, a device for viewing specially designed games in stereoscopic 3D using a disc that spins in front of the eyes\cite{zachara:stereo}; and the Light Pen, which allowed the user to draw directly onto the screen in art and music games. Although the Vectrex was following the trend of the other console manufacturers by releasing accessories, its peripherals were unique and innovative in comparison to keyboards and alternative controllers\cite{montfort:beam}. Unfortunately, both peripherals released for the Vectrex had incredibly short lives due to the crash, so it is difficult to gauge whether people would have perceived them as an exciting new piece of technology or simply a gimmick.



\section*{Conclusion}
The Vectrex competed using innovation, but that was not enough to save it from the crash. Its slim profit margin combined with the poor timing of its release hindered its success, while its differences may have made some people wary of purchasing it. Perhaps it would have fared better if the climate in the industry was more stable, as it could be argued that the Vectrex was never given a chance to develop and mature. If the crash wasn't impending shortly after its release, the Vectrex's innovations may instead have excited people.


\bibliographystyle{ieeetr}
\bibliography{game-platform-history-references}


\end{document}
